\documentclass[a4j]{jarticle}
\date{}
\usepackage[dvipdfmx]{graphicx}
\usepackage{longtable}
\usepackage{lscape}
\usepackage{color}
\usepackage{scalefnt}
\usepackage{listings,jlisting}
\lstset{	% ソースコード埋め込みの設定
	basicstyle={\small},%ソースコードの文字を小さくする
	commentstyle={\small\itshape},%コメントアウトの文字を小さくする
	frame=single,	% 枠線は単線
	tabsize=4,		% タブのサイズは4
	breaklines=true	% 自動改行
}
\bigin{document}

%表題
\makeatletter %ここから\makeatotherまで触らなくていい
	\def\@thesis{2019年度 東邦大学理学部情報科学科}
	\def\id#1{\def\@id{#1}}
	\def\department#1{\def\@department{#1}}
	
	\def\@maketitle{
		\begin{center}
			\vspace{10mm}
			{\large \@thesis \par}	%修士論文と記載される部分
			\vspace{50mm}
			{\huge\bf \@title \par}	% 論文のタイトル部分
			\vspace{15mm}
			{\Large 学籍番号 \@id \par}	% 学籍番号部分
			\vspace{5mm}
			{\Large \@author \par}	% 氏名
			\vspace{50mm}
		\end{center}
		\begin{flushright}
			{\large 金岡研究室}
		\end{flushright}
	}
\makeatother


\title{Abeの\\論文練習}
\id{5517001}
\author{阿部 衛}
\maketitle{\title}
\thispagestyle{empty}
\newpage 

\section{これが章作成}
章作成はこんな感じ
\subsection{これが節}
この節では、
画像をはる、表を作成することにしてみる
